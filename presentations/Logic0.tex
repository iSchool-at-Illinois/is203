\PassOptionsToPackage{unicode=true}{hyperref} % options for packages loaded elsewhere
\PassOptionsToPackage{hyphens}{url}
%
\documentclass[
  ignorenonframetext,
]{beamer}
\usepackage{pgfpages}
\setbeamertemplate{caption}[numbered]
\setbeamertemplate{caption label separator}{: }
\setbeamercolor{caption name}{fg=normal text.fg}
\beamertemplatenavigationsymbolsempty
% Prevent slide breaks in the middle of a paragraph:
\widowpenalties 1 10000
\raggedbottom
\setbeamertemplate{part page}{
  \centering
  \begin{beamercolorbox}[sep=16pt,center]{part title}
    \usebeamerfont{part title}\insertpart\par
  \end{beamercolorbox}
}
\setbeamertemplate{section page}{
  \centering
  \begin{beamercolorbox}[sep=12pt,center]{part title}
    \usebeamerfont{section title}\insertsection\par
  \end{beamercolorbox}
}
\setbeamertemplate{subsection page}{
  \centering
  \begin{beamercolorbox}[sep=8pt,center]{part title}
    \usebeamerfont{subsection title}\insertsubsection\par
  \end{beamercolorbox}
}
\AtBeginPart{
  \frame{\partpage}
}
\AtBeginSection{
  \ifbibliography
  \else
    \frame{\sectionpage}
  \fi
}
\AtBeginSubsection{
  \frame{\subsectionpage}
}
\usepackage{lmodern}
\usepackage{amssymb,amsmath}
\usepackage{ifxetex,ifluatex}
\ifnum 0\ifxetex 1\fi\ifluatex 1\fi=0 % if pdftex
  \usepackage[T1]{fontenc}
  \usepackage[utf8]{inputenc}
  \usepackage{textcomp} % provides euro and other symbols
\else % if luatex or xelatex
  \usepackage{unicode-math}
  \defaultfontfeatures{Scale=MatchLowercase}
  \defaultfontfeatures[\rmfamily]{Ligatures=TeX,Scale=1}
\fi
% use upquote if available, for straight quotes in verbatim environments
\IfFileExists{upquote.sty}{\usepackage{upquote}}{}
\IfFileExists{microtype.sty}{% use microtype if available
  \usepackage[]{microtype}
  \UseMicrotypeSet[protrusion]{basicmath} % disable protrusion for tt fonts
}{}
\makeatletter
\let\@@magyar@captionfix\relax
\@ifundefined{KOMAClassName}{% if non-KOMA class
  \IfFileExists{parskip.sty}{%
    \usepackage{parskip}
  }{% else
    \setlength{\parindent}{0pt}
    \setlength{\parskip}{6pt plus 2pt minus 1pt}}
}{% if KOMA class
  \KOMAoptions{parskip=half}}
\makeatother
\usepackage{xcolor}
\IfFileExists{xurl.sty}{\usepackage{xurl}}{} % add URL line breaks if available
\IfFileExists{bookmark.sty}{\usepackage{bookmark}}{\usepackage{hyperref}}
\hypersetup{
  pdftitle={Propositional Logic},
  pdfauthor={Dave Dubin},
  pdfborder={0 0 0},
  breaklinks=true}
\urlstyle{same}  % don't use monospace font for urls
\newif\ifbibliography
\usepackage{longtable,booktabs}
\usepackage{caption}
% These lines are needed to make table captions work with longtable:
\makeatletter
\def\fnum@table{\tablename~\thetable}
\makeatother
\setlength{\emergencystretch}{3em}  % prevent overfull lines
\providecommand{\tightlist}{%
  \setlength{\itemsep}{0pt}\setlength{\parskip}{0pt}}
\setcounter{secnumdepth}{-2}

% set default figure placement to htbp
\makeatletter
\def\fps@figure{htbp}
\makeatother

\usepackage[utf8]{inputenc}
\usepackage{amssymb}
\usepackage{syllogism}
\usepackage{mathtools}

\title{Propositional Logic}
\author{Dave Dubin}
\date{Fall Semester, 2020}

\begin{document}
\frame{\titlepage}

\begin{frame}{Reasoning and representation}
\protect\hypertarget{reasoning-and-representation}{}

\begin{itemize}[<+->]
\tightlist
\item
  With this topic we'll give some attention to logical reasoning and
  inference.
\item
  The same rules that govern valid argumentation also enable information
  systems to answer interesting questions.
\item
  Logic is also a means to describe a \emph{domain} (some part of the
  world we're interested in).
\end{itemize}

\end{frame}

\begin{frame}{Differences in notation}
\protect\hypertarget{differences-in-notation}{}

\begin{itemize}[<+->]
\tightlist
\item
  Both our authors explain how to construct logical expressions.
\item
  Magnus uses `\&' for `and' while Wallis uses \(\wedge\). I'll use
  \(\wedge\).
\item
  Wallis uses `\textasciitilde{}' for `not' while Magnus uses \(\neg\).
  I'll use \(\neg\).
\item
  Wallis uses lower case letters to represent propositions, while Magnus
  uses lower case. We'll use lower case letters.
\item
  Wallis only uses parentheses to disambiguate the scope of an operator.
\item
  Magnus insists on WFF rules that call for parentheses even where
  there's no ambiguity.
\item
  We will follow Magnus's example and add parentheses according to the
  WFF rules.
\end{itemize}

\end{frame}

\begin{frame}{Differences in vocabulary}
\protect\hypertarget{differences-in-vocabulary}{}

\begin{itemize}[<+->]
\tightlist
\item
  Magnus says his capital letters represent `sentences'
\item
  Wallis says a proposition is `a statement that has a truth value.'
\item
  I say that propositions are language independent bearers of truth
  values.
\item
  Think of them as the information content of simple declarative
  sentences.
\end{itemize}

\end{frame}

\begin{frame}{Different sentence, same proposition}
\protect\hypertarget{different-sentence-same-proposition}{}

\begin{enumerate}[<+->]
\tightlist
\item
  ``This table is made of wood.''
\item
  ``This is a wooden table.''
\item
  ``Ĉi tiu tablo estas el ligno.''
\item
  ``Jen ligna tablo.''
\item
  ``Dieser Tisch ist aus Holz.''
\item
  ``Dies ist ein Holztisch.''
\end{enumerate}

\end{frame}

\begin{frame}{Propositions, statements, sentences, and truth values}
\protect\hypertarget{propositions-statements-sentences-and-truth-values}{}

\begin{itemize}[<+->]
\item
  There are two truth values: \textbf{true} and \textbf{false}.
\item
  Propositions are the bearers of truth values: the kinds of things that
  can be true or false.
\item
  States of affairs are arrangements of reality that make propositions
  true or false.
\item
  We will model states of affairs using sets of proposition letters.
\end{itemize}

\end{frame}

\begin{frame}{Propositional logic}
\protect\hypertarget{propositional-logic}{}

\begin{itemize}[<+->]
\tightlist
\item
  We represent propositions with lower case letters (typically \(p, q,\)
  and \(r\)).
\item
  A set of proposition letters generates a set of states of affairs:
  ways the world might be.
\item
  \(n\) proposition letters generate \(2^n\) states of affairs.
\item
  \(\{pqr,pq\overline{r},p\overline{q}r,p\overline{qr},\overline{p}qr,\overline{p}q\overline{r},\overline{pq}r,\overline{pqr}\}\)
\end{itemize}

\end{frame}

\begin{frame}{Logical Operators}
\protect\hypertarget{logical-operators}{}

\begin{longtable}[]{@{}lcr@{}}
\toprule
Symbol & In natural language & Technical name\tabularnewline
\midrule
\endhead
\(\neg\) & not & negation\tabularnewline
\(\wedge\) & and & conjunction\tabularnewline
\(\vee\) & or & disjunction\tabularnewline
\(\rightarrow\) & if \ldots{} then & implication\tabularnewline
\(\leftrightarrow\) & if and only if & equivalence\tabularnewline
\bottomrule
\end{longtable}

\end{frame}

\begin{frame}{Logical Expressions}
\protect\hypertarget{logical-expressions}{}

\begin{itemize}[<+->]
\tightlist
\item
  A sentence constructed from proposition letters and operators is true
  or false in each state of affairs.
\item
  Consider, for example: \((({\neg}p \vee q) \rightarrow r)\)
\item
  The sentence is mapped to a truth value via the following tables
\end{itemize}

\end{frame}

\begin{frame}{Truth tables for the operators}
\protect\hypertarget{truth-tables-for-the-operators}{}

\begin{longtable}[]{@{}cc@{}}
\toprule
\(\varphi\) & \(\neg\varphi\)\tabularnewline
\midrule
\endhead
0 & 1\tabularnewline
1 & 0\tabularnewline
\bottomrule
\end{longtable}

\begin{longtable}[]{@{}cccccc@{}}
\toprule
\(\varphi\) & \(\psi\) & \(\varphi \wedge \psi\) & \(\varphi \vee \psi\)
& \(\varphi \rightarrow \psi\) &
\(\varphi \leftrightarrow \psi\)\tabularnewline
\midrule
\endhead
0 & 0 & 0 & 0 & 1 & 1\tabularnewline
0 & 1 & 0 & 1 & 1 & 0\tabularnewline
1 & 0 & 0 & 1 & 0 & 0\tabularnewline
1 & 1 & 1 & 1 & 1 & 1\tabularnewline
\bottomrule
\end{longtable}

\end{frame}

\begin{frame}{From natural language to propositional logic}
\protect\hypertarget{from-natural-language-to-propositional-logic}{}

\begin{itemize}[<+->]
\tightlist
\item
  ``I will only go to school if I get a cookie now.''
\item
  Compare with ``I will go to school if I get a cookie''
\item
  The latter is cookie implies school, but the former is school implies
  cookie.
\item
  You want \((s \rightarrow c)\)
\item
  Gloss: s = ``I will go to school'' and c = ``I get a cookie.''
\item
  John and Mary are running
\item
  \((j \wedge m)\)
\item
  A foreign national is entitled to social security if he has legal
  employment or if he has had such less than three years ago, unless he
  is currently also employed abroad.
\item
  \((((e \vee l) \wedge {\neg}a) \rightarrow s)\)
\item
  \(e =\) ``A foreign national has legal employment.''
\item
  \(l =\) ``He has had legal employment less than three years ago.''
\item
  \(a =\) ``He is currently employed abroad.''
\item
  \(s =\) ``He is entitled to social security.''
\end{itemize}

\end{frame}

\begin{frame}{Drawing truth tables for expressions}
\protect\hypertarget{drawing-truth-tables-for-expressions}{}

\begin{tabular}{@{ }c@{ }@{ }c@{ }@{ }c | c@{}@{}c@{}@{ }c@{ }@{ }c@{ }@{ }c@{ }@{ }c@{ }@{}c@{}@{ }c@{ }@{ }c@{ }@{}c@{ }}
p & q & r & ( & ( & $\neg$ & p & $\vee$ & q & ) & $\rightarrow$ & r & )\\
\hline 
1 & 1 & 1 &  &  & 0 & \textcolor{red}{1} & 1 & \textcolor{red}{1} &  & 1 & \textcolor{red}{1} & \\
1 & 1 & 0 &  &  & 0 & \textcolor{red}{1} & 1 & \textcolor{red}{1} &  & 0 & \textcolor{red}{0} & \\
1 & 0 & 1 &  &  & 0 & \textcolor{red}{1} & 0 & \textcolor{red}{0} &  & 1 & \textcolor{red}{1} & \\
1 & 0 & 0 &  &  & 0 & \textcolor{red}{1} & 0 & \textcolor{red}{0} &  & 1 & \textcolor{red}{0} & \\
0 & 1 & 1 &  &  & 1 & \textcolor{red}{0} & 1 & \textcolor{red}{1} &  & 1 & \textcolor{red}{1} & \\
0 & 1 & 0 &  &  & 1 & \textcolor{red}{0} & 1 & \textcolor{red}{1} &  & 0 & \textcolor{red}{0} & \\
0 & 0 & 1 &  &  & 1 & \textcolor{red}{0} & 1 & \textcolor{red}{0} &  & 1 & \textcolor{red}{1} & \\
0 & 0 & 0 &  &  & 1 & \textcolor{red}{0} & 1 & \textcolor{red}{0} &  & 0 & \textcolor{red}{0} & \\
\end{tabular}

\end{frame}

\begin{frame}{Drawing truth tables for expressions}
\protect\hypertarget{drawing-truth-tables-for-expressions-1}{}

\begin{tabular}{@{ }c@{ }@{ }c@{ }@{ }c | c@{}@{}c@{}@{ }c@{ }@{ }c@{ }@{ }c@{ }@{ }c@{ }@{}c@{}@{ }c@{ }@{ }c@{ }@{}c@{ }}
p & q & r & ( & ( & $\neg$ & p & $\vee$ & q & ) & $\rightarrow$ & r & )\\
\hline 
1 & 1 & 1 &  &  & \textcolor{red}{0} & \textcolor{green}{1} & 1 & 1 &  & 1 & 1 & \\
1 & 1 & 0 &  &  & \textcolor{red}{0} & \textcolor{green}{1} & 1 & 1 &  & 0 & 0 & \\
1 & 0 & 1 &  &  & \textcolor{red}{0} & \textcolor{green}{1} & 0 & 0 &  & 1 & 1 & \\
1 & 0 & 0 &  &  & \textcolor{red}{0} & \textcolor{green}{1} & 0 & 0 &  & 1 & 0 & \\
0 & 1 & 1 &  &  & \textcolor{red}{1} & \textcolor{green}{0} & 1 & 1 &  & 1 & 1 & \\
0 & 1 & 0 &  &  & \textcolor{red}{1} & \textcolor{green}{0} & 1 & 1 &  & 0 & 0 & \\
0 & 0 & 1 &  &  & \textcolor{red}{1} & \textcolor{green}{0} & 1 & 0 &  & 1 & 1 & \\
0 & 0 & 0 &  &  & \textcolor{red}{1} & \textcolor{green}{0} & 1 & 0 &  & 0 & 0 & \\
\end{tabular}

\end{frame}

\begin{frame}{Drawing truth tables for expressions}
\protect\hypertarget{drawing-truth-tables-for-expressions-2}{}

\begin{tabular}{@{ }c@{ }@{ }c@{ }@{ }c | c@{}@{}c@{}@{ }c@{ }@{ }c@{ }@{ }c@{ }@{ }c@{ }@{}c@{}@{ }c@{ }@{ }c@{ }@{}c@{ }}
p & q & r & ( & ( & $\neg$ & p & $\vee$ & q & ) & $\rightarrow$ & r & )\\
\hline 
1 & 1 & 1 &  &  & \textcolor{green}{0} & 1 & \textcolor{red}{1} & \textcolor{green}{1} &  & 1 & 1 & \\
1 & 1 & 0 &  &  & \textcolor{green}{0} & 1 & \textcolor{red}{1} & \textcolor{green}{1} &  & 0 & 0 & \\
1 & 0 & 1 &  &  & \textcolor{green}{0} & 1 & \textcolor{red}{0} & \textcolor{green}{0} &  & 1 & 1 & \\
1 & 0 & 0 &  &  & \textcolor{green}{0} & 1 & \textcolor{red}{0} & \textcolor{green}{0} &  & 1 & 0 & \\
0 & 1 & 1 &  &  & \textcolor{green}{1} & 0 & \textcolor{red}{1} & \textcolor{green}{1} &  & 1 & 1 & \\
0 & 1 & 0 &  &  & \textcolor{green}{1} & 0 & \textcolor{red}{1} & \textcolor{green}{1} &  & 0 & 0 & \\
0 & 0 & 1 &  &  & \textcolor{green}{1} & 0 & \textcolor{red}{1} & \textcolor{green}{0} &  & 1 & 1 & \\
0 & 0 & 0 &  &  & \textcolor{green}{1} & 0 & \textcolor{red}{1} & \textcolor{green}{0} &  & 0 & 0 & \\
\end{tabular}

\end{frame}

\begin{frame}{Drawing truth tables for expressions}
\protect\hypertarget{drawing-truth-tables-for-expressions-3}{}

\begin{tabular}{@{ }c@{ }@{ }c@{ }@{ }c | c@{}@{}c@{}@{ }c@{ }@{ }c@{ }@{ }c@{ }@{ }c@{ }@{}c@{}@{ }c@{ }@{ }c@{ }@{}c@{ }}
p & q & r & ( & ( & $\neg$ & p & $\vee$ & q & ) & $\rightarrow$ & r & )\\
\hline 
1 & 1 & 1 &  &  & 0 & 1 & \textcolor{green}{1} & 1 &  & \textcolor{red}{1} & \textcolor{green}{1} & \\
1 & 1 & 0 &  &  & 0 & 1 & \textcolor{green}{1} & 1 &  & \textcolor{red}{0} & \textcolor{green}{0} & \\
1 & 0 & 1 &  &  & 0 & 1 & \textcolor{green}{0} & 0 &  & \textcolor{red}{1} & \textcolor{green}{1} & \\
1 & 0 & 0 &  &  & 0 & 1 & \textcolor{green}{0} & 0 &  & \textcolor{red}{1} & \textcolor{green}{0} & \\
0 & 1 & 1 &  &  & 1 & 0 & \textcolor{green}{1} & 1 &  & \textcolor{red}{1} & \textcolor{green}{1} & \\
0 & 1 & 0 &  &  & 1 & 0 & \textcolor{green}{1} & 1 &  & \textcolor{red}{0} & \textcolor{green}{0} & \\
0 & 0 & 1 &  &  & 1 & 0 & \textcolor{green}{1} & 0 &  & \textcolor{red}{1} & \textcolor{green}{1} & \\
0 & 0 & 0 &  &  & 1 & 0 & \textcolor{green}{1} & 0 &  & \textcolor{red}{0} & \textcolor{green}{0} & \\
\end{tabular}

\end{frame}

\begin{frame}{Grammar of propositional logic}
\protect\hypertarget{grammar-of-propositional-logic}{}

Let \(P\) be a set of proposition letters and let \(p \in P\).

The following expression defines the recursive grammar for a logical
expression \(\varphi\) in
\href{https://en.wikipedia.org/wiki/Backus\%E2\%80\%93Naur_Form}{Backus--Naur
Form}:

\(\varphi \Coloneqq p|\neg\varphi|(\varphi \wedge \varphi)|(\varphi \vee \varphi)|(\varphi \rightarrow \varphi)|(\varphi \leftrightarrow \varphi)\)

\end{frame}

\begin{frame}{Syntactically conforming expressions}
\protect\hypertarget{syntactically-conforming-expressions}{}

Let \(P = \{o,q,r,s\}\)

Examples of grammatically conforming expressions include:

\begin{itemize}[<+->]
\tightlist
\item
  \(r\)
\item
  \({\neg}q\)
\item
  \((s \leftrightarrow o)\)
\item
  \(({\neg}(s \leftrightarrow \neg\neg{\neg}o) \rightarrow (q \wedge q))\)
\end{itemize}

Grammatically \emph{incorrect} expressions would include:

\begin{itemize}[<+->]
\tightlist
\item
  \(\neg\vee p\)
\item
  \(\vee ) p \neg\)
\item
  \({\neg}p \vee q \rightarrow r\)
\end{itemize}

How many correct expressions are consistent with the last one?

\end{frame}

\begin{frame}{Syntactic ambiguity}
\protect\hypertarget{syntactic-ambiguity}{}

These conforming expressions are all consistent with
\({\neg}p \vee q \rightarrow r\)

\begin{itemize}[<+->]
\tightlist
\item
  \((({\neg}p \vee q) \rightarrow r)\)
\item
  \(({\neg}(p \vee q) \rightarrow r)\)
\item
  \({\neg}((p \vee q) \rightarrow r)\)
\item
  \(({\neg}p \vee (q \rightarrow r))\)
\item
  \({\neg}(p \vee (q \rightarrow r))\)
\end{itemize}

\end{frame}

\begin{frame}{Well-formed formulas}
\protect\hypertarget{well-formed-formulas}{}

Which of the following are formulas in propositional logic?

\begin{itemize}[<+->]
\tightlist
\item
  \(p \rightarrow {\neg}q\)
\item
  \({\neg}{\neg} \wedge q \vee p\)
\item
  \(p{\neg}q\)
\end{itemize}

\end{frame}

\begin{frame}{Logical truth and logical falsity.}
\protect\hypertarget{logical-truth-and-logical-falsity.}{}

\begin{itemize}[<+->]
\item
  A statement \(\varphi\) is logically true if it is true in every state
  of affairs generated by its propositional variables.
\item
  A statement \(\varphi\) is logically false if it is false in every
  state of affairs generated by its propositional variables.
\item
  If a statement \(\varphi\) is neither logically true or logically
  false then it is contingent.
\item
  Examples:

  \begin{enumerate}[<+->]
  \tightlist
  \item
    \((q \vee {\neg}q)\) is logically true.
  \item
    \((q \wedge {\neg}q)\) is logically false.
  \end{enumerate}
\end{itemize}

\end{frame}

\begin{frame}{Consistency}
\protect\hypertarget{consistency}{}

\begin{itemize}[<+->]
\item
  A set of propositional logic statements is consistent if at least one
  state of affairs satisfies every statement in the set.
\item
  A set of propositional logic statements is inconsistent if no state of
  affairs satisfies every statement in the set.
\end{itemize}

\end{frame}

\begin{frame}{Inference and validity}
\protect\hypertarget{inference-and-validity}{}

\begin{itemize}[<+->]
\tightlist
\item
  A conclusion is \emph{valid} with respect to a set of premises if the
  conclusion is true in every sitation where the premises are true.
\item
  One can validly infer a conclusion \(\varphi\) from a set of premises
  \(P\) if the negation of \(\varphi\) is inconsistent with the set of
  statements \(P\).
\end{itemize}

\end{frame}

\begin{frame}{Computation and expressive power}
\protect\hypertarget{computation-and-expressive-power}{}

\begin{enumerate}[<+->]
\tightlist
\item
  Computing a truth value for a formula takes linear time.
\item
  Computing a truth table for validity takes exponential time.
\item
  The problem of testing for validity in propositional logic is
  decidable: there exists a mechanical method that computes the answer,
  at least in principle.
\end{enumerate}

\end{frame}

\end{document}
